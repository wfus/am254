%If you start a line with a "percent" symbol (like %), then that line is a "comment" and won't show up in your actual document.  

%Every document starts with a documentclass. 
\documentclass[11pt]{article}
\usepackage[margin=0.8in]{geometry}

%After that, it's useful to list an author and give a title.
\author{William Fu}
\title{Problem Set 1}

%Graphicx is used to include pictures in LaTeX files. 
\usepackage{graphicx}

%The AMS packages. They contain a lot of useful math-related goodies.
\usepackage{amsthm}
\usepackage{amsmath}
\usepackage{amsfonts}

%The below command makes sure that eve ry section starts on a new page. That way if you have a new section for every CA, they'll all print out on separate pieces of paper. 
\usepackage{titlesec}

%The amsthm package lets you format different types of mathematical ideas nicely. You use it by defining "\newtheorem"s as below:
\newtheorem{problem}{Problem}
\newtheorem{theorem}{Theorem}
\newtheorem*{proposition}{Proposition}
\newtheorem{lemma}[theorem]{Lemma}
\newtheorem{corollary}[theorem]{Corollary}
\theoremstyle{definition}
\newtheorem{defn}[theorem]{Definition}

%The "\newcommand" command lets you specify a custom command. This should be used wisely to add semantic meaning to otherwise confusing sequences of commands - not just speed up typing. (If you want suggestions for shortcuts you can ask Thayer).
%Here is an example definition of a bra and ket from Quantum Mechanics.
\newcommand{\bra}[1]{\langle #1 |}
\newcommand{\ket}[1]{| #1 \rangle}
\newcommand{\range}{\text{ range }}
\newcommand{\im}{\text{im }}


%Adding your name here lets you make sure every page has your name, so that your psets don't get mixed up.
\usepackage{fancyhdr}
\pagestyle{fancy}
\lhead{William Fu}
\rhead{Problem Set 1}

%Everything above here is just commands which don't create any main-document text directly.
%You have to put all your writing within \begin{document} and \end{document} clauses.
\begin{document}


\section{Exercise 1}

\begin{enumerate}
\item 
We will begin from the RHS. First, note that from the definition of expectation, we have
\[ \langle H(\sigma) \rangle = \sum_\sigma P_\beta(\sigma) H(\sigma) \]
Then, expanding the RHS out,
\begin{align*}
	\langle H(\sigma) \rangle - \frac{1}{\beta}S(\beta) 
	&=  \sum_\sigma P_\beta(\sigma) H(\sigma) + \frac{1}{\beta} \sum_\sigma P_\beta(\sigma) \log P_\beta(\sigma) \\
	&= - \frac{\partial}{\partial \beta} \langle H(\sigma) \rangle \\
	&= \sum_\sigma P_\beta(\sigma)\left( H(\sigma) + \frac{1}{\beta} \log P_\beta(\sigma) \right) \\
	&= \sum_\sigma P_\beta(\sigma)\left( H(\sigma) + \frac{1}{\beta}(-\log Z(\beta) - \beta H(\sigma) )\right) \\
	&= \frac{1}{\beta}\left(-\log Z(\beta)\right) \sum_\sigma P_\beta(\sigma) \\
	&= -\frac{1}{\beta} \log Z(\beta) \\
	&= F(\beta)
\end{align*}

\item
	Note that in the previous problem we already showed that $\langle H(\sigma) \rangle = \frac{\partial}{\partial \beta }[\beta F(\beta)]$. Then, taking the derivative of both sides again with respect to $\beta$,
\begin{align*}
	- \frac{\partial^2}{\partial \beta^2}[\beta F(\beta)]
	&= -\frac{Z'(\beta)}{Z(\beta)} \\
	&= -\frac{\frac{d}{d\beta} \sum_x e^{-\beta H(x)}}{Z(\beta)} \\
	&= \frac{\sum_x H(x) e^{-\beta H(x)}}{Z(\beta)} \\
	&= \langle H(x) \rangle
\end{align*}

\item 
	As always, we will start from the right hand side.
	\begin{align*}
		\beta^2 \frac{\partial F(\beta)}{\partial \beta}
		&= \beta^2 \left(\frac{\log (Z(\beta ))}{\beta ^2}-\frac{Z'(\beta )}{\beta  Z(\beta )}\right) \\
		&= \log Z(\beta) - \beta \frac{Z'(\beta)}{Z(\beta)} \\
		&= - \beta F(\beta) - \beta \frac{Z'(\beta)}{Z(\beta)} 
	\end{align*}
However, note that from the last problem, we had
\[ \langle H(\sigma) \rangle = \frac{\partial}{\partial \beta}[\beta F(\beta)] = -\frac{Z'(\beta)}{Z(\beta)}\]
Making this substitution to the original expression, and using the identity from part (a),
\[ \beta^2\frac{\partial F(\beta)}{\partial \beta} = -\beta F(\beta) + \beta \langle H (\sigma) \rangle
= -\beta\left( \langle H(\sigma) \rangle - \frac{1}{\beta} S(\beta) \right)  + \beta \langle H (\sigma) \rangle
= S(\beta)\]
which is what we wanted.

\item 
	We will show this

\item
	Replacing the $\beta$ with $T$, we have the function
	\[ F(T) = -T \log Z(1/T)\]
	To prove that $F(T)$ is nonincreasing, it suffices to show that $ T \log Z(1/T)$ is nondecreasing. Note that $T$ is increasing, so we must just show that the partition function $Z(1/T)$ is nondecreasing.
	As the number of thermodynamically accessible states, the partition function should increase as the temperature increases. Therefore, $T \log Z(1/T)$ is nondecreasing, and $F(T)$ is nonincreasing. To show that the function is concave, it suffices to show that the derivative with respect to $T$ is strictly decreasing. Note that
	\[ \frac{dF(T)}{dT} = -\log Z(1/T) - T \frac{1}{Z(1/T)}(Z'(1/T))(-1/T^2)  \]

\end{enumerate}

\newpage
\section{Exercise 2}

\textbf{(I used one of the recommended texts for this question: Merhav Page 84)}

First, we will want to find the partition function as a function of $\beta$. For notational convenience, let $\sigma_N =\sigma_0$ so our states wrap around periodically. Then,
\[Z(\beta) = 
\sum_{\mathbf{\sigma}} \exp\left(\beta\sum_i^N \sigma_i \sigma_{i+1} + \beta h\sum_i^N\sigma_i \right)\]
As per the hint, we want to decompose this into $2 \times 2$ matrices, so we will make sure that we have two different $\sigma_i$ in the expression for both terms.
\[Z(\beta) = 
\sum_{\mathbf{\sigma}} \exp\left(\beta\sum_i^N \sigma_i \sigma_{i+1} + \beta \frac{h}{2}\sum_i^N (\sigma_i + \sigma_{i+1}) \right)\]
Now, we know that $\sigma_i \in \{-1, +1\}$, two states. Let us identify the states as vectors instead,
\[
	\sigma_i = +1 = \begin{pmatrix} 1 \\ 0 \end{pmatrix}
	~ ~ ~ ~ ~ ~ ~ ~
	\sigma_i = -1 = \begin{pmatrix} 0 \\ 1 \end{pmatrix}
\]
Then, we can write our exponent in matrix form.
\[ \exp\left(\beta\sum_i^N \sigma_i \sigma_{i+1} + \beta \frac{h}{2}\sum_i^N (\sigma_i + \sigma_{i+1}) \right) = \sigma_i^T P \sigma_{i+1}\]
where we can construct $P$ by just considering all of the $4$ cases for the values of $\sigma_i, \sigma_{i+1}$. 
\[ P = 
	\begin{pmatrix}
		e^{\beta + \beta h} & e^{-\beta} \\
		e^{-\beta} & e^{\beta - \beta h}
	\end{pmatrix}
\]
Then, when we iterate through all possibilities of the states of $\sigma_1, \dots, \sigma_N$, note that the terms $\sigma_i^T \sigma_i = 1$. Thus, we have
\[  \sum_{\sigma_1, \dots, \sigma_N} \sigma_i^T P \sigma_i = \sum \sigma_1^T P (\sigma_2\sigma_2^T) P \cdots (\sigma_{N-1} \sigma_{N-1}^T) P \sigma_1 = \sum \sigma_1^T P^N \sigma_1 \]
Again, since we are taking the possibilities of $\sigma_1 \in \{ (1, 0)^T, (0, 1)^T\}$, this is the same as getting the top left or bottom right element of $P^N$, which is equivalent to $\lambda_1^N + \lambda_2^N$ for $\lambda_1, \lambda_2$ eigenvalues of $P$. The eigenvalues (using mathematica) are:
\[ \lambda_1 =\frac{1}{2} e^{\beta  (-(h+1))} \left(e^{2 \beta }+\sqrt{e^{4 \beta }+4 e^{2 \beta  h}+e^{4 \beta  (h+1)}-2 e^{2 \beta  (h+2)}}+e^{2 \beta  (h+1)}\right) \]
\[ \lambda_2 =\frac{1}{2} e^{\beta  (-(h+1))} \left(e^{2 \beta }-\sqrt{e^{4 \beta }+4 e^{2 \beta  h}+e^{4 \beta  (h+1)}-2 e^{2 \beta  (h+2)}}+e^{2 \beta  (h+1)}\right) \]
\begin{enumerate}
\item 
Now that we have a expression for the partition function we can evaluate the free energy density. For notational convenience, let $Z_N(\beta, h) = \lambda_1^N + \lambda_2^N$, where $\lambda_1, \lambda_2$ are the eigenvalues as defined above. THen,
\begin{align*}
	f(\beta) 
	&= -\lim_{N \to \infty} \frac{1}{\beta N} \log Z_N(\beta) \\
	&= -\lim_{N \to \infty} \frac{1}{\beta N} \log (\lambda_1^N + \lambda_2^N)
\end{align*}
Then, assuming that $\lambda_1 > \lambda_2$, we can approximate the limit by throwing away the $\lambda_2$ term in the log since the exponent of $N\to \infty$ makes $\lambda_1^N \gg \lambda_2^N$. 
\begin{align*}
	f(\beta) 
	&= -\lim_{N \to \infty} \frac{1}{\beta N }\log (\lambda_1^N) \\
	&= \frac{1}{\beta}\log \lambda_1
\end{align*}
Since we have already expressed $\lambda_1$ in terms of only $\beta$ and $h$, we have what we wanted.
\[ \boxed{f(\beta) \approx  \frac{1}{2\beta} e^{\beta  (-(h+1))} \left(e^{2 \beta }+\sqrt{e^{4 \beta }+4 e^{2 \beta  h}+e^{4 \beta  (h+1)}-2 e^{2 \beta  (h+2)}}+e^{2 \beta  (h+1)}\right) }\]

\item

\item
\end{enumerate}


\newpage
\section{Exercise 3}





\end{document}
